\documentclass[%
 reprint,
%superscriptaddress,
%groupedaddress,
%unsortedaddress,
%runinaddress,
%frontmatterverbose, 
%preprint,
%preprintnumbers,
%nofootinbib,
%nobibnotes,
%bibnotes,
 amsmath,amssymb,
 aps,
%pra,
%prb,
%rmp,
%prstab,
%prstper,
%floatfix,
]{revtex4-2}

\usepackage{graphicx}% Include figure files
\usepackage{dcolumn}% Align table columns on decimal point
\usepackage{bm}% bold math
%\usepackage{hyperref}% add hypertext capabilities
%\usepackage[mathlines]{lineno}% Enable numbering of text and display math
%\linenumbers\relax % Commence numbering lines

%\usepackage[showframe,%Uncomment any one of the following lines to test 
%%scale=0.7, marginratio={1:1, 2:3}, ignoreall,% default settings
%%text={7in,10in},centering,
%%margin=1.5in,
%%total={6.5in,8.75in}, top=1.2in, left=0.9in, includefoot,
%%height=10in,a5paper,hmargin={3cm,0.8in},
%]{geometry}
\newcommand{\greekfi}{Greek.fi }


\begin{document}

\title{\greekfi: Decentralized American-style Options Protocol}

\author{Mahmoud Lababidi}
\email{ml@greek.fi}
\affiliation{%
\greekfi Protocol\\
Decentralized Options Protocol
}

\date{June 26, 2025}

\begin{abstract}
\greekfi is a protocol that enables decentralized American-style exercise options that are fully collateralized without requiring oracles or margin. 
It is designed for composability, exercisability, and univsersal use on any EVM compatible chain and allows any ERC20 token to be used as collateral and consideration ( i.e. WBTC, WETH, stETH, USDC, USDT) 

Similar to yield generating tokens, the protocol generates two ERC20 tokens: a redemption token (which provides capital efficiency) and an option token.
We dive into the menchanics of these tokens and how they impact a new level of composibility to create a new options ecosystem that both, encompasses and surpasses traditional financial options strategies.


\end{abstract}

\maketitle

\section{\label{sec:introduction}Introduction}

\subsection{Overview}
Options in the crypto have exploded in usage and approaches. Currently, there's over \$6B USD in 24hour volume in perpetuals trading. 
Additionally, Deribit was purchased by Coinbase for nearly \$3B USD. 
This is of no surprise as the traditional finance market sees \$2.7T USD in daily notional trading in the options market (\$600B in open interest daily) the crypto market can easily catch up to these numbers.

One issue arises in the current state of the options industry in the DeFi space: early exercizable (American) options have not existed in a decentralized, tokenized manner.
American options need the following abilities to be considered as such:
\begin{itemize}
  \item Expirable at some future fixed date
  \item exercizable so that the option holder can swap consideration for underlying collateral prior to expiration
  \item The option writer can use the collateral freely after expiration, unless...
  \item The contract written has been exercised, providing consideration to the option writer to be used freely
\end{itemize}

In order to achieve the goal of providing an American options protocol in a fully decentralized manner, all of the above needs to be solved in a decentralized manner.
We present the \greekfi Options Protol, the first fully decentralized, collateralized, tokenized, exercizable, expirable options protocol in the Ethereum ecosystem.
The protocol achieves the following:
\begin{itemize}
  \item Fully ERC20 compatible - allowing full composability
  \item Fully collateralized protocol so that every option can be exercised to swap for the collateral
  \item Exercizable so that the option holder can swap consideration tokens for the collateral
  \item The option writer can redeem consideration prior to expiration, or
  \item The option writer can redeem collateral post expiration, or
  \item The option writer can redeem collateral prior to expiration if they also own the option
  \item Two tokens/contracts drive the protocol (LONG and SHORT) similar to yield platforms
  \item Oracle free, margin free
\end{itemize}

This protocol will unlock a trove of possibilities to DeFi, for example:

\begin{itemize}
  \item hedge risk
  \item take on risk
  \item leverage
  \item earn yield through covered calls
  \item transfer options from within ecosytems to other ecosystems
\end{itemize}


\greekfi is a decentralized American-style options protocol that allows for the creation of options with full collateralization. 
The design of the protocol along with full collateralization allows exercisability, composability, and universal use on any EVM compatible chain.

\subsection{Background}

Options in the crypto space have grown.
With the purchase of Deribit by Coinbase, the options ecosystem has 

\greekfi is a decentralized American style options protocol that
allows for the creation of options with full collateralization.

Typically, options have been centralized, with providers like Deribit, 
being the most popular. These providers use margin to allow a variety of
options to be created and traded. Additionally, most derivative
strategies involve perpetual futures, which require oracles for pricing.

Our protocol is designed to be a decentralized alternative to these
providers. No oracles are required, and no margin is required.
Additionally, any locked collateral used to create one side of the trade
can be used as collateral for loans, as the position is represented by
an ERC20 token.

One limitation in all the different ecosystems is the ability to connect them and take them to other chains.

\section{\label{sec:protocol}Protocol Overview}

The \greekfi protocol is a contract factory system that creates a new pair of contracts for each tuple that represents an options 
(Collateral, Consideration, Strike, Expiration, CallOrPut). This allows the protocol to be upgradeable with new features to improve useability. 
The pair of contracts created are two ERC20 contracts that are coupled to each other, LONG and SHORT. 
By depositing collateral tokens into the protocol, one mints the two tokens allowing them to be used in different strategies.
Let's desccribe the tokens in detail to illustrate their usability. 

\subsubsection{LONG Token}

The LONG token represents the position of the option buyer, they are
entitled to purchase the underlying collateral asset when the option is
exercised.
\begin{itemize}
\item The LONG token can allow you to exercise the option at any
time before expiration.
\item The LONG token can be traded on DEXs like
Uniswap, CoW, and RFQs like 0x, Bebop.
\end{itemize}

\subsubsection{SHORT Token}

The short token represents the position of the option writer, they are
obligated to swap the underlying collateral asset for the consideration
asset when the LONG option is exercised.
\begin{itemize}
\item The SHORT token can be used
to redeem the underlying asset after expiration.
\item The SHORT token can
be used as collateral for loans (Silo).
\item Together, the LONG and SHORT
tokens can be redeemed for the underlying asset before expiration to
recover the collateral (Neutral position).
\end{itemize}


Our protocol is simple. Any ERC20 token can be used as collateral for
options. This includes WBTC, WETH, stETH, SBTC, AAVE, UNI, and more. The
user chooses the strike price, expiration, and option type (call or put)
when creating the option. A user mints two ERC20 tokens when creating an
option:
\begin{itemize}
\item \textbf{LONG}: Represents the right to exercise the option
\item \textbf{SHORT}: Represents the obligation and right to the collateral
\end{itemize}

Use any EVM compatible chain to mint the options.

\subsection{Token Functionality}


\subsection{Technical Implementation}

\subsubsection{Token Linking}

The two tokens/contracts are linked to each other. 

\subsubsection{Margin-Free Design}

Because the LONG and SHORT tokens are linked to each other, the protocol
is able to provide a margin-free experience.

\subsubsection{Collateralization}

The protocol is able to provide full collateralization for the options.


\subsubsection{Compound Options}

The protocol is able to provide compound options. This allows for the creation of options with multiple underlying assets.


\subsubsection{Loanable Collateral}

The protocol is able to provide loanable collateral for the options.


\subsubsection{Composibility}

The protocol is able to be composed with other protocols to create more complex strategies.


\subsubsection{Default Swaps}

The protocol is able to provide default swaps for the options.

OPTION INSTEAD OF TOKENS DROP

\subsection{Example: Minting Options}

\begin{enumerate}
\item
  Connect your wallet to the protocol
\item
  Select option parameters:
  \begin{itemize}
  \item
    Collateral Asset: WETH
  \item
    Consideration Asset: USDC
  \item
    Strike Price: 5000 USDC
  \item
    Expiration Date: 30 days
  \item
    Option Type: Call
  \end{itemize}
\item
  Mint Approve the LONG contract to capture your WETH
\item
  Receive LONG and SHORT tokens
\end{enumerate}

\section{\label{sec:trading}Trading}

Once options are minted, the obvious question is how to trade them. We
have considered using a DEX AMM, such as Uniswap, but options could have
low volume and low liquidity. This can cause slippage and nobody wants
that.

The obvious solution is to use RFQs, such as 0x, Bebop, etc. This allows
for the options to be traded at a price that is determined by the
market. This would allow partnering with market makers to provide
liquidity for the options. This would also solve pricing the options
where the MMs set their own prices. This will create a new market maker
ecosystem.

\subsection{Short Token Trading}

The SHORT token represents two things:
Collateral/consideration ownership and a short position in the option
trade. Let's say you sold an in the money call (WETH at 3000 expiring
soon). When you sold your LONG token you received a premium of 1000 USDC
in addition to IV and Time Value. This is potentially priced in the
SHORT token making it tradable since it has a value.

\section{\label{sec:vaults}Vaults}

Vaults allow for more complex strategies using the options protocol.
This simplifies the experience for users and reduces the amount of work
required to implement these strategies.

\subsection{Covered Call Vaults}

One strategy, such as covered call vaults, which allow for the creation
of options with a portion of the collateral as the option premium. A
simple strategy could involve users depositing WETH as collateral, and
then the vault handles:
\begin{enumerate}
\item minting options with the collateral
\item selling the options
\item let the options expire worthless
\item redeem the collateral
\item mint more options with the collateral and repeat
\end{enumerate}

This strategy is nearly identical to ETF strategies (i.e., XYLD). Covered
call vaults do not exist on chain, but this protocol allows for the
creation of these strategies. This same strategy can be applied for
Covered Put vaults, where the vault mints the put options and sells
them.

\subsection{Margined Options Vaults}

Similar to covered call vaults, margined options vaults allow for the
minting of call options using collateral. Afterwards, the vault takes a
loan on the collateral (AAVE, Silo, etc) and uses the proceeds to mint
more options. This allows for the vault to mint more options than it
would have been able to otherwise.

\section{\label{sec:faq}FAQ}

\subsection{Traditional vs. On-Chain Options}

\paragraph{In the traditional covered call world, I don't need to mint
tokens or anything. Why do I need to mint tokens here?}

In traditional call strategies, the custodian (E-Trade, Fidelity, etc)
holds the collateral and establishes the short position. On Chain, we do
not have custodians. We have smart contracts. We need to mint tokens to
represent the two positions, which technically is neutral wrt to the
option trade.

\paragraph{Why do this on chain?}

You don't have to. If you're looking for high frequency trading, this is
not for you. If you're looking for a low-friction, low-cost way to
create covered options strategies, this is for you.

\paragraph{Ok, I'm holding X, why do this? couldn't I just get a loan on
the X and buy more X?}

Again, you don't have to, but this is almost like staking.

\subsection{Technical Questions}

\paragraph{Explain this SHORT token? isn't that just a PUT?}

The short token is simply a combination of two things: Your ownership of
the collateral + the short position in the option trade. The SHORT token
is basically looking at your brokerage account and seeing ``1 wETH + (-1
call option)''. If you tack on a LONG token it's like seeing ``1 wETH +
(-1 call option) + 1 call option'' which equals ``1 wETH''.

\paragraph{How do you do this for PUTs?}

Yes! Believe it or not, puts are simply calls where the consideration and
collateral are swapped. For example, in a put use USDC becomes the
collateral and WETH is the consideration. What this means is that you
have the option to ``buy'' USDC with your WETH.

The only real difference between puts and calls is the strike price. In
a call you buy 1 wETH for 4000 USDC, so in a put instead of saying ``buy
USDC for 0.00025 wETH/USDC'' you say ``sell you 1 wETH for 4000
USDC/wETH''.

\subsection{User Experience}

\paragraph{I want to buy some options, how do I do that?}

Use the trade page to buy options. It uses 0x and Bebop to provide
liquidity and execute trades.

\paragraph{Okay, I'm up on my options, how do I sell them?}

Use the trade page to sell options, as well as buy.

\paragraph{Can I try this on Testnets?}

Yes, definitely. Choose testnets when you mint. We'll give you free wETH
testnet tokens to mint with.

\section{\label{sec:appendix}Appendix}

\subsection{Key Concepts}

\begin{itemize}
\item
  \textbf{Call Option}: Right to buy an asset at a specified price
  before expiration
\item
  \textbf{Put Option}: Right to sell an asset at a specified price
  before expiration
\item
  \textbf{Strike Price}: The price at which the option can be exercised
\item
  \textbf{Expiration Date}: The last date the option can be exercised
\item
  \textbf{LONG Token}: Represents the right to exercise the option
\item
  \textbf{SHORT Token}: Represents the obligation and right to the
  collateral
\end{itemize}

\begin{acknowledgments}
We wish to acknowledge the support of the DeFi community in developing
Greek.fi, offering suggestions and encouragement, testing new versions,
and providing valuable feedback on the protocol design and implementation.
\end{acknowledgments}

\end{document}
